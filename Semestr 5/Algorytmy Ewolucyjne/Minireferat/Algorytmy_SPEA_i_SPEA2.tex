\documentclass{beamer}

\usepackage[T1]{fontenc}
\usepackage[polish]{babel}

\title[SPEA, SPEA2]{Algorytmy optymalizacji wielokryterialnej \newline SPEA i SPEA2}
\author[Jakub Kuciński]{Jakub Kuciński}
\usepackage{datetime}
\newdate{date}{31}{01}{2021}
\date{\displaydate{date}}

\usetheme{Warsaw}

\let\babellll\lll
\let\lll\relax
\usepackage{amsthm} %proof
\usepackage{amsmath}
\usepackage{amssymb}

\usepackage{colortbl}
\usepackage{tikz}
\usetikzlibrary{matrix}
\newcommand{\szary}{\cellcolor{gray}}
\usepackage{array,tikz}
\usetikzlibrary{tikzmark,arrows.meta}
\usepackage{mathtools}
\usepackage[noend]{algpseudocode}
\usepackage{textcomp}
\usepackage{algorithm}
\usepackage{algorithmicx}

\begin{document}

\frame{\titlepage}


\begin{frame}{Front Pareto}
\begin{figure}[htp]
    \centering
    \includegraphics[width=6cm]{pareto_2.png}
    \includegraphics[width=6cm]{Dominated-non-dominated-and-Pareto-front-solution-set.png}
\end{figure}
\end{frame}


\begin{frame}{SPEA}
$\boldsymbol{SPEA}(N,\overline{N})$:\smallskip\\
$\boldsymbol{P_0}$ = Initial-Population()\\
$\boldsymbol{\overline{P}_0}$ = $\emptyset$, $\boldsymbol{t}$ = 0\\
\textbf{while not} Termination-Condition():\\
\quad \quad $\boldsymbol{\overline{P}_{t+1}}$ = Update-archive($\boldsymbol{P_t}, \boldsymbol{\overline{P}_t}$)\\
\quad \quad Fitness-assignment($\boldsymbol{P_t}, \boldsymbol{\overline{P}_t}$) \\
\quad \quad $\boldsymbol{{P}_{t+1}}$ = Selection($\boldsymbol{P_t}, \boldsymbol{\overline{P}_t}$)\\
\quad \quad Recombination($\boldsymbol{{P}_{t+1}}$)\\
\quad \quad Mutation($\boldsymbol{{P}_{t+1}}$)\\
\quad \quad Increment($\boldsymbol{t}$)\\
\textbf{return} $\boldsymbol{\overline{P}_t}$\\
\bigskip
Oznaczenia:\\
$\boldsymbol{N,\overline{N}}$ -- rozmiar populacji, maksymalny rozmiar archiwum\\
$\boldsymbol{{P}_t}$ -- populacja w $\boldsymbol{t}$-tej iteracji\\
$\boldsymbol{\overline{P}_t}$ -- archiwum w $\boldsymbol{t}$-tej iteracji\\
\end{frame}


\begin{frame}{Update-archive}
Uaktualnianie archiwum:
\begin{itemize}
    \item Skopiuj archiwum $\boldsymbol{\overline{P}_t}$ do $\boldsymbol{\overline{P}'}$.
    \item Skopiuj osobniki niezdominowane w $\boldsymbol{{P}_t}$ do $\boldsymbol{\overline{P}'}$.
    \item Usuń z $\boldsymbol{\overline{P}'}$ osobniki zdominowane.
    \item Jeśli liczba osobników w zbiorze $\boldsymbol{\overline{P}'}$ przekracza $\boldsymbol{\overline{N}}$ to zmniejsz ich liczbę przez Clustering.
    \item Przypisz $\boldsymbol{\overline{P}'}$ do $\boldsymbol{\overline{P}_{t+1}}$.
\end{itemize}
\end{frame}


\begin{frame}{Clustering}
\begin{itemize}
    \item Zainicjalizuj $\boldsymbol{C} = \bigcup_{i\in\boldsymbol{\overline{P}'}} \{\{ i \}\}$.
    \item Jeśli $\vert \boldsymbol{C} \vert \leq \overline{N}$ to przeskocz do ostatniego punktu.
    \item Policz odległość (w przestrzeni funkcji celu) między wszystkimi klastrami $c1, c2 \in \boldsymbol{C}$:
    $$d_c = \frac{1}{\vert c_1 \vert \vert c_2 \vert}\cdot  \sum_{\substack{i_1\in c_1 \\ i_2 \in c_2}} d(i_1, i_2)$$
    \item Połącz dwa klastry znajdujące się najbliżej siebie. Powróć do drugiego punktu.
    \item Każdy klaster zastąp jego centroidem. Zwróć zbiór tych centroidów.
\end{itemize}
\end{frame}

\begin{frame}{Ilustracja Clusteringu}
\begin{figure}[htp]
    \centering
    \includegraphics[width=10cm]{ClusterinExplanation.png}
    \caption{Eckart Zitzler 1999. Diss. ETH No. 13398. Evolutionary Algorithms for Multiobjective Optimization: Methods and Applications}
\end{figure}
\end{frame}


\begin{frame}{Fitness-assignment}
Obliczanie przystosowania $F(i)$ osobników:
\begin{itemize}
    \item Najpierw każdemu osobnikowi $i \in$ $\boldsymbol{\overline{P}_t}$ przypisywana jest wartość $S(i) \in [0, 1)$ proporcjonalna do liczby dominowanych przez nią osobników z $\boldsymbol{{P}_t}$. Formalniej:
    $$S(i) = \frac{\vert\{ \,j\;\vert\; j\in \boldsymbol{{P}_t} \wedge i \succ j\} \vert}{N+1}$$
    Przystosowanie $i$ jest równe $S(i)$.
    \item Przystosowanie osobnika $j \in$ $\boldsymbol{{P}_t}$ wylicza się przez sumowanie wartości $S$ wszystkich osobników z $\boldsymbol{\overline{P}_t}$, które dominują $j$:
    $$F(j) = 1 +  \sum_{\substack{i\in \boldsymbol{\overline{P}_t} \\ i\, \succ j}} S(i)$$
    Dodanie jedynki zapewnia, że osobniki z $\boldsymbol{\overline{P}_t}$ będą miały lepsze przystosowanie od osobników z $\boldsymbol{{P}_t}$.
\end{itemize}
\end{frame}


\begin{frame}{Komentarz do przystosowania}
\begin{figure}[htp]
    \centering
    \includegraphics[width=10cm]{FitnessExplanation.png}
    \caption{Eckart Zitzler 1999. Diss. ETH No. 13398. Evolutionary Algorithms for Multiobjective Optimization: Methods and Applications}
\end{figure}
W SPEA jako algorytm selekcji został zaproponowany binarny turniej na sumie zbiorów $\boldsymbol{{P}_t}$ i $\boldsymbol{\overline{P}_t}$.
\end{frame}


\begin{frame}{Problemy SPEA}
\begin{itemize}
    \item Osobniki z populacji, które są dominowane przez te same osobniki z archiwum mają taką samą wartość przystosowania (w szczególności jeśli w archiwum jest jeden osobnik to cała populacja ma to samo przystosowanie).
    \item Jeśli dużo osobników w populacji jest nieporównywalnych, to mamy bardzo mało informacji z dominowania. SPEA nie wykorzystuje informacji o gęstości osobników w populacji.
    \item Clustering co prawda pozwala zachować charakterystykę frontu w archiwum, ale obcina brzegowe osobniki.
\end{itemize}
\end{frame}


\begin{frame}{SPEA2}
$\boldsymbol{SPEA2}(N,\overline{N})$:\smallskip\\
$\boldsymbol{P_0}$ = Initial-Population()\\
$\boldsymbol{\overline{P}_0}$ = $\emptyset$, $\boldsymbol{t}$ = 0\\
\textbf{while not} Termination-Condition():\\
\quad \quad Fitness-assignment($\boldsymbol{P_t}, \boldsymbol{\overline{P}_t}$) \\
\quad \quad $\boldsymbol{\overline{P}_{t+1}}$ = Update-archive($\boldsymbol{P_t}, \boldsymbol{\overline{P}_t}$)\\
\quad \quad $\boldsymbol{{P}_{t+1}}$ = Selection($\boldsymbol{\overline{P}_{t+1}}$)\\
\quad \quad Recombination($\boldsymbol{{P}_{t+1}}$)\\
\quad \quad Mutation($\boldsymbol{{P}_{t+1}}$)\\
\quad \quad Increment($\boldsymbol{t}$)\\
\textbf{return} $\boldsymbol{\overline{P}_t}$\\
\end{frame}


\begin{frame}{Fitness-assignment}
\begin{itemize}
    \item Każdemu osobnikowi $\boldsymbol{i}$ z $\boldsymbol{P_t}$ oraz $\boldsymbol{\overline{P}_t}$ przypisujemy wartość $S(i) = \vert\{ \,j\;\vert\; j\in \boldsymbol{\overline{P}_t} \cup \boldsymbol{P_t} \wedge i \succ j\} \vert$ odpowiadającą liczbie osobników, które dominuje.
    \item Dla każdego osobnika obliczane jest surowe przystosowanie:
    $$R(i) = \sum_{\substack{j\in \boldsymbol{\overline{P}_t} \cup \boldsymbol{P_t} \\ j\, \succ i}} S(j)$$
    \item Dla każdego osobnika $\boldsymbol{i}$ obliczana jest gęstość $\boldsymbol{\sigma_i^k}$ równa odległości do $\boldsymbol{k}$-tego najbliższego osobnika (w przestrzeni funkcji celu). Zazwyczaj przyjmuje się $\boldsymbol{k} = \sqrt{\boldsymbol{N+\overline{N}}}$. Gęstość określa wzór:
    $$D(i) = \frac{1}{\sigma_i^k+2}$$
    \item Końcowe przystosowanie osobnika określa wzór: 
    $$F(i) = R(i) + D(i)$$
\end{itemize}
\end{frame}

\begin{frame}{Update-archive}
    \begin{itemize}
        \item Najpierw kopiowane zostają niezdominowane osobniki z populacji i archiwum do nowego archiwum: $\boldsymbol{\overline{P}_{t+1}} = \{ i \; \vert \; i \in \boldsymbol{P_t} \cup \boldsymbol{\overline{P}_t} \wedge F(i) < 1 \}$.
        \item Jeśli $\vert \boldsymbol{\overline{P}_{t+1}} \vert < \boldsymbol{\overline{N}}$ to do $\boldsymbol{\overline{P}_{t+1}}$ zostają dodane najlepsze według przystosowania $\boldsymbol{\overline{N}} - \vert \boldsymbol{\overline{P}_{t+1}} \vert$ osobniki.
        \item Jeśli $\vert \boldsymbol{\overline{P}_{t+1}} \vert > \boldsymbol{\overline{N}}$ to iteracyjnie usuwane zostają osobniki z  $\boldsymbol{\overline{P}_{t+1}}$ do momentu, aż $\vert \boldsymbol{\overline{P}_{t+1}} \vert = \boldsymbol{\overline{N}}$. Procedura odcinania osobnika polega na wyborze takiego osobnika, który jest w najmniejszej odległości do innego osobnika. Dla osobników o równej minimalnej odległości porównywana jest odległość do drugiego najbliższego itd. Formalniej wybierany jest taki osobnik $i$, że $\forall_{j \in \boldsymbol{\overline{P}_{t+1}}} i \leq_d j$, gdzie:
        \begin{align*}
            i \leq_d j \Leftrightarrow \; & \forall_{0<k<\vert \boldsymbol{\overline{P}_{t+1}} \vert} \; \sigma_i^k = \sigma_j^k \; \vee \\
            & \exists_{0<k<\vert \boldsymbol{\overline{P}_{t+1}} \vert} \; \left[\left(\forall_{0<l<k} \; \sigma_i^l = \sigma_j^l \right) \wedge \sigma_i^k < \sigma_j^k\right]
        \end{align*}
    \end{itemize}
\end{frame}


\begin{frame}{Ilustracja odcinania}
    \begin{figure}[htp]
    \centering
    \includegraphics[width=10cm]{TruncationExplanation.png}
    \caption{Zitzler, Eckart; Laumanns, Marco; Thiele, Lothar 2001. SPEA2: Improving the strength pareto evolutionary algorithm}
\end{figure}
W SPEA2 algorytm selekcji to binarny turniej na zbiorze $\boldsymbol{\overline{P}_{t+1}}$.
\end{frame}


\begin{frame}{Funkcje ciągłe}
\begin{figure}[htp]
    \centering
    \includegraphics[width=8cm]{QVfunction.png}
    \includegraphics[width=10cm]{QVperformance.png}
    \caption{Zitzler, Eckart; Laumanns, Marco; Thiele, Lothar 2001. SPEA2: Improving the strength pareto evolutionary algorithm}
\end{figure}
\end{frame}


\begin{frame}{Funkcje ciągłe cd.}
\begin{figure}[htp]
    \centering
    \includegraphics[width=8cm]{KURfunction.png}
    \includegraphics[width=10cm]{KURperformance.png}
    \caption{Zitzler, Eckart; Laumanns, Marco; Thiele, Lothar 2001. SPEA2: Improving the strength pareto evolutionary algorithm}
\end{figure}
\end{frame}


\begin{frame}{Problem plecakowy dla m=3}
\begin{figure}[htp]
    \centering
    \includegraphics[width=8cm]{Knapsackfunction.png}
    \includegraphics[width=10cm]{KP3.png}
    \caption{Zitzler, Eckart; Laumanns, Marco; Thiele, Lothar 2001. SPEA2: Improving the strength pareto evolutionary algorithm}
\end{figure}
\end{frame}


\begin{frame}{Problem plecakowy dla m=4}
\begin{figure}[htp]
    \centering
    \includegraphics[width=8cm]{Knapsackfunction.png}
    \includegraphics[width=10cm]{KP4.png}
    \caption{Zitzler, Eckart; Laumanns, Marco; Thiele, Lothar 2001. SPEA2: Improving the strength pareto evolutionary algorithm}
\end{figure}
\end{frame}


\begin{frame}{Wyniki testów}
\begin{itemize}
    \item SPEA2 daje znacząco lepsze wyniki od SPEA
    \item SPEA2 ma bardzo podobną wydajność do NSGA2
    \item NSGA2 znajduje szerszy front Pareto od SPEA2
    \item SPEA2 zwraca bardziej równomiernie rozmieszczone osobniki od NSGA2
\end{itemize}
\end{frame}



\begin{frame}{Źródła}
\begin{thebibliography}{9}
\bibitem{1999}
Eckart Zitzler (1999). Diss. ETH No. 13398. \textit{Evolutionary Algorithms for Multiobjective Optimization: Methods and Applications}
\bibitem{2001} 
Zitzler, Eckart; Laumanns, Marco; Thiele, Lothar (2001). \textit{SPEA2: Improving the strength pareto evolutionary algorithm}
\end{thebibliography}
\end{frame}


\end{document}