\documentclass{article}

\usepackage{polski}
\usepackage[utf8]{inputenc}

\usepackage{fancyhdr} % Required for custom headers
\usepackage{lastpage} % Required to determine the last page for the footer
\usepackage{extramarks} % Required for headers and footers
\usepackage[usenames,dvipsnames]{color} % Required for custom colors
\usepackage{graphicx} % Required to insert images
\usepackage{listings} % Required for insertion of code
\usepackage{courier} % Required for the courier font
\usepackage{lipsum} 
\usepackage{amsfonts}
\usepackage{hyperref}
\usepackage{tikz}
\usepackage{amsmath}
\usepackage{pdfpages}
\usepackage{mathtools}
\usepackage{float}
\usepackage{amsthm}
 
\usepackage[noend]{algpseudocode}
\usepackage{algorithm}
\usepackage{algorithmicx}

\newtheorem{thm}{Twierdzenie}
\newtheorem{remark}{Uwaga}
\newtheorem{lemat}{Lemat}
\newtheorem{wniosek}{Wniosek}
\newtheorem{definicja}{Definicja}
\newtheorem{ciekawostka}{Ciekawostka}
\newtheorem{przyklad}{Przykład}

\newenvironment{rozw}{\paragraph{Rozwiązanie:}}{\hfill}


% Margins
\topmargin=-0.45in
\evensidemargin=0in
\oddsidemargin=0in
\textwidth=6.6in
\textheight=9.0in
\headsep=0.25in

\linespread{1.1} % Line spacing


\title{Zadanie 4 lista 7}

\begin{document}

\maketitle

\tableofcontents

\section{Rozwiązanie}
Będziemy utrzymywać następujący niezmiennik:
\begin{itemize}
    \item W korzeniu każdego drzewa znajduje się jeden kredyt
    \item Wierzchołek wewnętrzny ma przynajmniej 2 kredyty, jeśli utracił syna.
\end{itemize}

\subsection{Insert}
Operacja insert polega na utworzeniu nowego drzewa z pojedynczą wartością i dodanie go na koniec listy drzew. Operacje utworzenia i dodania na koniec listy wymagają jednego kredytu. Żeby utrzymać niezmiennik musimy dodatkowo zostawić jeden kredyt w korzeniu dodanego drzewa. Zatem operacja insert wymaga 2 kredytów.

\subsection{Meld}
Operacja meld polega na połączeniu dwóch list zawierających drzewa obu kopców i wybraniu wskaźnika odpowiadającego mniejszemu z minimów w obu kopcach. Potrzebujemy na to 1 kredyt.

\subsection{Findmin}
Operacja findmin polega na odczytaniu wartości najmniejszego elementu w kopcu, co możemy wykonać patrząc na wskaźnik MIN w kopcu. Wymaga to 1 kredytu.

\subsection{Decreasekey}
Operacji decreasekey będziemy przydzielać 4 jednostki kredytowe. Jeśli operacja zmniejszenia wartości klucza nie zaburzy porządku kopcowego to potrzebujemy tylko 1 kredytu na wykonanie operacji zmniejszenia wartości. Jeśli natomiast zaburza porządek kopcowy oraz ojciec tego wierzchołka nie stracił jeszcze 2 synów to odcinamy poddrzewo o korzeniu w tym wierzchołku, dodajemy je na koniec listy drzew i zostawiamy kredyt w korzeniu dodanego drzewa. Wymaga to 2 kredytów. Pozostałe 2 kredyty zostawiamy w ojcu odciętego wierzchołka. Pozostaje rozważyć przypadek, gdy decreasekey zaburza porządek kopcowy oraz ojciec tego wierzchołka stracił już dwóch synów. Odcinamy wówczas poddrzewo o korzeniu w ojcu o ile dziadek nie stracił jeszcze 2 synów (jeśli stracił to postępujemy analogicznie do bieżącego przypadku schodząc w stronę korzenia). Zauważmy, że skoro ojciec stracił już 2 synów, to oba odcięcia odłożyły w nim kredyty. Można wykorzystać je do odcięcia poddrzewa, dodania go do listy oraz odłożenia kredytu w korzeniu nowego drzewa - 2 kredyty. Pozostałe kredyty odkładamy w dziadku. Zatem na operacje decreasekey wystarczą 4 kredyty.

\subsection{Deletemin}
Operacja deletemin polega na usunięciu minimalnej wartości w kopcu, dodania na koniec listy poddrzew o korzeniach w synach usuniętego wierzchołka oraz połączeniu drzew znajdujących się na liście tak, aby nie istniały dwa o tym samym rzędzie (łączymy ze sobą tylko drzewa tych samych rzędów). W celu analizy wymaganej do przydzielenia dla deletemin liczby kredytów udowodnimy najpierw następujący lemat:
\begin{lemat}
Dla każdego wierzchołka $v$ kopca Fibonacciego (w których kaskadowe wykonanie operacji cut wykonywane jest dopiero wtedy, gdy wierzchołek traci trzeciego syna) o rzędzie k, drzewo zakorzenione w $v$ ma rozmiar wykładniczy względem k.
\end{lemat}
\begin{proof}
Weźmy dowolny wierzchołek $v$ kopca. Przez $y_1, \ldots, y_k$ oznaczmy synów $v$ uporządkowanych w kolejności ich podłączania. Rozważmy wierzchołek $y_i$. Skoro wierzchołki $y_1,\ldots, y_{i-1}$ zostały podwieszone przed $y_i$ to w momencie podwieszania $y_i$, rząd $v$ musiał wynosić przynajmniej $i-1$. Łączymy tylko drzewa równych rzędów, stąd w momencie podwieszania rząd $y_i$ wynosił tyle co rząd $v$, czyli wynosił co najmniej $i-1$. Jaki rząd może mieć teraz $y_i$? Mógł zmniejszyć się co najwyżej o 2, bo pozwalamy tylko na stratę 2 synów w każdym wierzchołku. Zatem obecny rząd $y_i$ wynosi co najmniej $i-3$. Wiemy więc, że w każdym momencie dla dowolnego wierzchołka $v$ jego i-ty syn ma co najmniej rząd $i-3$.

Niech $F_k$ będzie najmniejszym drzewem spełniającym powyższe zależności (korzeń ma rząd k). Skoro $F_k$ jest najmniejsze, to każdy syn korzenia jest najmniejszym możliwym poddrzewem, a skoro i-ty syn ma rząd co najmniej $i-3$ to w przypadku $F_k$ rząd i-tego syna będzie wynosił $\max \{0, i-3\}$. Skoro poddrzewo też jest minimalne, to poddrzewo o korzeniu w i-tym synie jest drzewem $F_{\max \{0, i-3\}}$. Stąd dla drzewa $F_k$ jego synowie są korzeniami drzew $F_0, F_0, F_0, F_1, F_2, \ldots, F_{k-4}, F_{k-3}$. Zatem liczba wierzchołków $|F_k|$ jest równa $$1 + 2F_0 + \sum_{i=0}^{k-3} |F_i| \geq Fib(k-1)$$ 
Wiemy, że $Fib(k-1) \sim \left(\frac{1+\sqrt{5}}{2}\right)^{k-1}$, stąd liczba wierzchołków w drzewie o rzędzie $k$ jest wykładnicza względem $k$.
\end{proof}
Z powyższego lematu wnioskujemy, że każdy wierzchołek ma stopień co najwyżej $\mathcal{O}(\log{n})$.\\
Dla operacji deletemin należy przydzielić $\mathcal{O}(\log{n})$ kredytów. Po usunięciu minimum potrzebujemy $\mathcal{O}(\log{n})$ kredytów na podłączenie drzew o korzeniach w synach do listy drzew kopca oraz $\mathcal{O}(\log{n})$ kredytów na odłożenie w ich korzeniach -- bo rząd usuniętego wierzchołka wynosił co najwyżej $\mathcal{O}(\log{n})$. Podczas przeglądania listy drzew, w celu złączenia drzew o tym samym rzędzie, koszt odwiedzenia dołączanego drzewa oraz dołączenia go do drugiego może zostać pokryty przez kredyt znajdujący się w korzeniu. Koszt odwiedzenia drzew, które nie zostaną podłączone do innego drzewa można opłacić używając $\mathcal{O}(\log{n})$ kredytów -- każdy wierzchołek jest stopnia co najwyżej $\mathcal{O}(\log{n})$, więc mamy co najwyżej $\mathcal{O}(\log{n})$ możliwych różnych stopni drzew w kopcu, a na koniec na liście nie mogą występować dwa drzewa o tym samym stopniu. Stąd ostatecznie operacja deletemin wymaga $\mathcal{O}(\log{n})$ kredytów.


\end{document}