\documentclass{article}

\usepackage{polski}
\usepackage[utf8]{inputenc}

\usepackage{fancyhdr} % Required for custom headers
\usepackage{lastpage} % Required to determine the last page for the footer
\usepackage{extramarks} % Required for headers and footers
\usepackage[usenames,dvipsnames]{color} % Required for custom colors
\usepackage{graphicx} % Required to insert images
\usepackage{listings} % Required for insertion of code
\usepackage{courier} % Required for the courier font
\usepackage{lipsum} 
\usepackage{amsfonts}
\usepackage{hyperref}
\usepackage{tikz}
\usepackage{amsmath}
\usepackage{pdfpages}
\usepackage{mathtools}
\usepackage{float}
\usepackage{amsthm}
 
\usepackage{algorithmic}

\newtheorem{thm}{Twierdzenie}
\newtheorem{remark}{Uwaga}
\newtheorem{lemat}{Lemat}
\newtheorem{wniosek}{Wniosek}
\newtheorem{definicja}{Definicja}
\newtheorem{ciekawostka}{Ciekawostka}
\newtheorem{przyklad}{Przykład}

\newenvironment{rozw}{\paragraph{Rozwiązanie:}}{\hfill}


% Margins
\topmargin=-0.45in
\evensidemargin=0in
\oddsidemargin=0in
\textwidth=6.6in
\textheight=9.0in
\headsep=0.25in

\linespread{1.1} % Line spacing


\title{Zadanie 4 lista 2}

\begin{document}

\maketitle

\tableofcontents

\section{Wstępne spostrzeżenia}
Chcemy tak pokolorować wierzchołki, by na każdej ścieżce było co najwyżej k pokolorowanych wierzchołków. Zauważmy, że dla dowolnej ścieżki $S_1$ istnieje ścieżka $S_2$ o końcach w pewnych liściach tego drzewa, zatem możemy ograniczyć się do rozważania ścieżek o końcach w liściach. \\
Zastanówmy się jakie wierzchołki chcemy wybierać do pokolorowania. Rozsądne wydaje się kolorowanie wierzchołków, które występują w najmniejszej liczbie ścieżek. Takimi wierzchołkami są oczywiście liście, bo jeśli ścieżka ma zawierać dany liść, to któryś z jej końców musi być w tym liściu. 

\section{Idea rozwiązania}
Dla danego drzewa $G_1$ i dopuszczalnej liczby $k_1$ pokolorowanych wierzchołków na ścieżce, o ile $k_1>1$ to kolorujemy wszystkie liście tego drzewa. Po takim kolorowaniu wszystkie ścieżki w drzewie G mają co najwyżej 2 pokolorowane wierzchołki - pokolorowane są tylko liście, więc ścieżka musi się kończyć w liściu by zawierać dany pokolorowany wierzchołek, a skoro ma 2 końce, to stąd maksymalnie 2 pokolorowane wierzchołki. Następnie tworzymy nowe drzewo $G_2$ usuwając z drzewa $G_1$ wszystkie liście (oczywiście jest to drzewo, bo $G_1$ było drzewem, a usunięcie liści nie rozspójnia drzewa). Oznaczamy $k_2$ $\leftarrow$ $k_1 - 2$. Następnie postępujemy analogicznie jak z $G_1$ i $k_1$. Zauważmy, że w $G_1$ każda ścieżka zawiera maksymalnie 2 pokolorowane wierzchołki, więc po ponownym rozszerzeniu $G_2$ do $G_1$ otrzymamy ścieżki o maksymalnie 4 pokolorowanych wierzchołkach. Analogicznie tworzymy $G_i$ oraz $k_i$ do momentu, aż $k_i \leq 1$. W przypadku $k_i = 1$ możemy pokolorować jeszcze tylko jeden wierzchołek, bo pokolorowanie dwóch lub więcej utworzy nam ścieżkę o $k+1$ pokolorowanych wierzchołkach. Jeśli $k_i = 0$ to kończymy kolorowanie. Widzimy, że nasze kolorowanie daje pewne poprawne rozwiązanie. Później udowodnimy, że jest ono optymalne.

\newpage
\section{Algorytm}
\begin{algorithmic}
\WHILE{$k > 1$ oraz G jest niepuste}
    \STATE Pokoloruj wszystkie liście drzewa G
    \STATE Usuń wszystkie liście z drzewa G
    \STATE $k \leftarrow k-2$
\ENDWHILE
\IF{$k = 1$ i G jest niepuste}
    \STATE Pokoloruj dowolny wierzchołek
\ENDIF
\end{algorithmic}


\section{Dowód poprawności algorytmu}
Pokażemy, że dowolne optymalne kolorowanie można sprowadzić do kolorowania zwracanego przez nasz algorytm.

\begin{lemat}
Dla danego drzewa $G$ oraz $k=1$ można pokolorować co najwyżej 1 wierzchołek. Dla wygody zapisu kolorowanie spełniające warunki zadania dla danego $k$ będę nazywał $k$ kolorowaniem.
\end{lemat}
\begin{proof}
Załóżmy nie wprost, że istnieje drzewo, dla którego można pokolorować dwa wierzchołki. Weźmy takie drzewo i kolorowanie. Skoro graf jest drzewem, to jest spójny, równoważnie drogowo-spójny, czyli istnieje droga między pokolorowanymi wierzchołkami, czyli istnieje ścieżka o 2 pokolorowanych wierzchołkach. Sprzeczność.
\end{proof}

\begin{lemat}
Załóżmy, że $C_1$ dla danego $k_1$ jest optymalnym kolorowaniem drzewa $G_1$ oraz $C_1$ ma wszystkie liście pokolorowane. Wtedy dla drzewa $G_2$ powstałego w wyniku usunięcia wszystkich liści z $G_1$ oraz $k_2 = k_1 - 2$ kolorowanie $C_2$ powstałe w wyniku zawężenia kolorowania $C_1$ do zbioru wierzchołków $G_2$ jest optymalnym kolorowaniem. Ponadto dowolne optymalne kolorowanie drzewa $G_2$ można rozszerzyć do optymalnego kolorowania grafu $G_1$.
\end{lemat}
\begin{proof}
Załóżmy nie wprost, że $C_2$ nie jest optymalnym kolorowaniem grafu $G_2$. Niech $\overline{C_2}$ będzie jego dowolnym optymalnym kolorowaniem. Skoro $C_2$ nie jest optymalnym, to liczba pokolorowanych wierzchołków w $C_2$ jest mniejsza niż w $\overline{C_2}$. Rozważmy kolorowanie $\overline{C_1}$ powstałe w wyniku rozszerzenia $G_2$ do $G_1$ i pokolorowania jego liści. Oczywiście skoro $\overline{C_2}$ było optymalnym $k_2 = k_1 - 2$ kolorowaniem grafu $G_2$, to $\overline{C_1}$ jest optymalnym $k_1$ kolorowaniem grafu $G_1$ (dodanie pokolorowanych liści może zwiększyć liczbę pokolorowanych wierzchołków w dowolnej ścieżce o co najwyżej 2). Zatem $\overline{C_1}$ oraz $C_1$ są optymalnymi kolorowaniami $G_1$. Ale skoro $\overline{C_2}$ ma więcej pokolorowanych wierzchołków niż $C_2$ to $\overline{C_1}$ ma więcej niż $C_1$. Sprzeczność z optymalnością $C_1$.\\
Oczywiście każde optymalne kolorowanie grafu $G_2$ można sprowadzić do optymalnego kolorowania grafu $G_1$ dodając pokolorowane liście $G_2$.
\end{proof}

\begin{lemat}
Załóżmy, że $C$ jest optymalnym $k>1$ kolorowaniem drzewa $G$. Wtedy można sprowadzić $C$ do optymalnego kolorowania, które koloruje wszystkie liście $G$.
\end{lemat}
\begin{proof}
Pierwszym spostrzeżeniem jest, że liczba pokolorowanych wierzchołków jest większa równa liczbie liści, bo skoro $k>1$ to pokolorowanie wszystkich liści jest poprawnym kolorowaniem, więc optymalne nie może zawierać mniej wierzchołków. Jeśli $C$ zawiera wszystkie liście to teza jest spełniona. Załóżmy więc, że istnieje liść, który nie jest pokolorowany w $C$. Weźmy dowolny taki liść. Oznaczmy go jako $v$. Weźmy wierzchołek $w$ leżący najbliżej $v$, który nie jest liściem i jest pokolorowany (istnieje, bo liczba pokolorowanych $\geq$ liczba liści, a jeden liść jest niepokolorowany). Pokolorujmy $v$ i odkolorujmy $w$. Pokażemy, że nowe kolorowanie jest poprawne. Załóżmy nie wprost, że nie jest poprawne. Wtedy, skoro pokolorowaliśmy tylko jeden nowy wierzchołek, to ścieżka, która zawiera więcej niż k pokolorowanych wierzchołków musi zawierać $v$. Nazwijmy tę ścieżkę $S$. Wiemy również, że kolorowanie przed zmianą było poprawne, zatem S nie może zawierać $w$ (gdyby zawierało, to $S$ przed zmianą kolorowania też zawierałaby więcej niż k pokolorowanych wierzchołków, co jest sprzeczne z poprawnością $C$). Ale skoro $w$ jest najbliższym $v$ pokolorowanym w $C$ wierzchołkiem, to istnieje ścieżka z $w$ zawierająca wszystkie pokolorowane z $S$ wierzchołki i ma ona więcej niż k pokolorowanych wierzchołków. Sprzeczność. Zatem nowe kolorowanie jest poprawne, a skoro zawiera tyle samo wierzchołków co optymalne $C$, to również jest optymalne. Postępując analogicznie dla wszystkich niepomalowanych liści otrzymujemy optymalne malowanie, które maluje wszystkie liście.
\end{proof}

\begin{thm}
Każde optymalne kolorowanie można przekształcić do postaci zwracanej przez nasz algorytm.
\end{thm}
\begin{proof}
Indukcja względem $k$:
\begin{enumerate}
    \item $k \in \{0, 1\}$
    \begin{enumerate}
        \item $k = 0$ Trywialne
        \item $k = 1$ Z lematu 1.
    \end{enumerate}
    \item Załóżmy, że dla $k$ można. Pokażemy dla $k+2$.
    
    Weźmy dowolne optymalne $k+2$ kolorowanie $C$ drzewa $G$. Z lematu 3. możemy przekształcić je do optymalnego kolorowania, które ma wszystkie liście pokolorowane. Z kolei z lematu 2. wiemy, że możemy sprowadzić ten problem do $k$ kolorowania grafu $\overline{G}$ powstałego z usunięcia z $G$ wszystkich liści. Wiemy z lematu 2. że kolorowanie ${C}$ obcięte do wierzchołków z $\overline{G}$ jest optymalne dla $\overline{G}$. Z założenia indukcyjnego można je sprowadzić do postaci zwracanej przez nasz algorytm. Znowu z lematu 2. możemy je rozszerzyć do optymalnego kolorowania $G$ dodając z powrotem pokolorowane liście. Zatem dla $k+2$ nasze założenie również jest spełnione.
\end{enumerate}
Z 1. i 2. na mocy zasady indukcji matematycznej każde optymalne kolorowanie można przekształcić do postaci zwracanej przez nasz algorytm.
\end{proof}

\section{Złożoność pamięciowa i obliczeniowa}
\subsection{Złożoność pamięciowa}
Drzewo $G$ możemy trzymać w odwrotnej liście sąsiedztwa (lista ojcostwa). Odwrotna lista sąsiedztwa drzewa zajmuje $\mathcal{O}(n)$ pamięci (w ogólności $\mathcal{O}(n+m)$, ale dla drzewa $m = n -1$). Do sprawdzenie czy wierzchołek jest liściem możemy utworzyć tablicę pomocniczą $t$, która dla każdego wierzchołka będzie trzymała liczbę wychodzących z niego krawędzi - $\mathcal{O}(n)$ pamięci. Dodatkowo do trzymania zbioru liści możemy użyć kolejki, której rozmiar może wynosić co najwyżej tyle ile jest wierzchołków w drzewie, czyli $\mathcal{O}(n)$. Zatem łącznie potrzebujemy $\mathcal{O}(n)$ pamięci.
\subsection{Złożoność obliczeniowa}
Liście możemy trzymać w kolejce. W każdym obrocie pętli pobieramy liście z kolejki i kolorujemy je w czasie stałym. Następnie zmniejszamy liczbę wychodzących krawędzi dla ojca każdego liścia o 1 i jeśli wynosi 0 to dodajemy go do nowej kolejki (dzięki odwrotnej liście sąsiedztwa i tablicy $t$ wykonujemy to w czasie stałym). Zmniejszamy $k$ o 2 w czasie stałym. Wszystkie operacje wykonują się w czasie stałym. Należy zastanowić się zatem ile łącznie we wszystkich obrotach pętli będziemy pobierać wierzchołków z kolejek. To w dużej mierze zależy od struktury drzewa oraz wartości $k$, ale w pesymistycznym przypadku będziemy musieli pobrać wszystkie wierzchołki drzewa, czyli łącznie we wszystkich obrotach pętli wykonamy $\mathcal{O}(n)$ operacji. Wykonanie ostatniego warunku if wymaga stałego czasu, więc nie wpływa na złożoność asymptotyczną algorytmu.


\end{document}