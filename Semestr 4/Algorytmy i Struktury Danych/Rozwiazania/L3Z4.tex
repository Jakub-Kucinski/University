\documentclass{article}

\usepackage{polski}
\usepackage[utf8]{inputenc}

\usepackage{fancyhdr} % Required for custom headers
\usepackage{lastpage} % Required to determine the last page for the footer
\usepackage{extramarks} % Required for headers and footers
\usepackage[usenames,dvipsnames]{color} % Required for custom colors
\usepackage{graphicx} % Required to insert images
\usepackage{listings} % Required for insertion of code
\usepackage{courier} % Required for the courier font
\usepackage{lipsum} 
\usepackage{amsfonts}
\usepackage{hyperref}
\usepackage{tikz}
\usepackage{amsmath}
\usepackage{pdfpages}
\usepackage{mathtools}
\usepackage{float}
\usepackage{amsthm}
 
\usepackage[noend]{algpseudocode}
\usepackage{algorithm}
\usepackage{algorithmicx}

\newtheorem{thm}{Twierdzenie}
\newtheorem{remark}{Uwaga}
\newtheorem{lemat}{Lemat}
\newtheorem{wniosek}{Wniosek}
\newtheorem{definicja}{Definicja}
\newtheorem{ciekawostka}{Ciekawostka}
\newtheorem{przyklad}{Przykład}

\newenvironment{rozw}{\paragraph{Rozwiązanie:}}{\hfill}


% Margins
\topmargin=-0.45in
\evensidemargin=0in
\oddsidemargin=0in
\textwidth=6.6in
\textheight=9.0in
\headsep=0.25in

\linespread{1.1} % Line spacing


\title{Zadanie 4 lista 3}
\author{Jakub Kuciński, prowadzący Szymon Dudycz}

\begin{document}

\maketitle

\tableofcontents

\section{Treść zadania}
(\textcolor{green}{P}) Dane jest nieukorzenione drzewo z naturalnymi wagami na krawędziach oraz liczba naturalna C. Ułóż algorytm obliczający, ile jest par wierzchołków odległych od siebie o C.

\section{Idea rozwiązania}
Aby obliczyć liczbę par wierzchołków odległych od siebie o C można dla każdego wierzchołka obliczyć liczbę wierzchołków odległych od niego o C, następnie zsumować otrzymane wyniki i podzielić przez dwa -- każda para zostanie policzona dwa raz, raz dla każdego wierzchołka z pary. Skoro graf jest drzewem, to istnieje dokładnie jedna droga między dowolnymi dwoma wierzchołkami. Wiemy zatem, że jeśli będziemy przechodzili drzewo za pomocą np. DFS'a i zapamiętywali odległość w jakiej jesteśmy od początkowego wierzchołka, to odwiedzając dany wierzchołek wiemy w jakiej jest odległości od początkowego (bo istnieje tylko jedna droga). Jeśli uznajemy 0 za liczbę naturalną to osobno musimy rozpatrzyć przypadek $C = 0$, bo wówczas każdy wierzchołek jest w odległości 0 do samego siebie i każda para $(v, v)$ jest rozpatrywana dokładnie raz.

\newpage
\section{Algorytm}
\begin{algorithm}
\caption{$DFS(v, d)$}
\begin{algorithmic}[1]
\State $k \leftarrow 0$
\If{$d = C$}
    \State $k\leftarrow k + 1$
\EndIf
\For{$u \in N(v)$}
    \If{$u$ nieodwiedzony}
        \State oznacz $u$ jako odwiedzony
        \If{$d+d(v,u) \leq C$}
            \State $k\leftarrow k + DFS(u, d+d(v,u))$
        \EndIf
    \EndIf
\EndFor
\Return $k$
\end{algorithmic}
\end{algorithm}

\begin{algorithm}
\caption{$Algorytm$}
\begin{algorithmic}[1]
\State $p\leftarrow 0$
\For{$v \in V(G)$}
    \State oznacz wszystkie wierzchołki jako nieodwiedzone
    \State oznacz $v$ jako odwiedzony
    \State $p \leftarrow p + DFS(v, 0)$
\EndFor
\If{C = 0}
    \State $p \leftarrow p + n(G)$
\EndIf
\Return $p/2$
\end{algorithmic}
\end{algorithm}

\section{Złożoność obliczeniowa i pamięciowa}
\subsection{Złożoność pamięciowa}
Graf możemy przechowywać w postaci listy sąsiedztwa, ale oprócz sąsiadów, możemy dodatkowo trzymać również wagę krawędzi łączącej dane wierzchołki. Wymaga to $\mathcal{O}(n+m)$ pamięci, czyli dla drzewa $\mathcal{O}(n)$. Informacje o odwiedzonych wierzchołkach można trzymać w zwykłej tablicy, co wymaga $\mathcal{O}(n)$ pamięci. Pojedyncze wywołanie $DFS(v, d)$ wymaga stałej ilości pamięci stosu. Mamy jednak wywołania rekurencyjne, więc zużywana pamięć będzie wprost proporcjonalna do głębokości wywołań rekurencyjnych, a skoro mamy $n$ wierzchołków, to będzie ona wynosiła co najwyżej $\mathcal{O}(n)$. Zatem całkowita złożoność pamięciowa algorytmu wynosi $\mathcal{O}(n)$.

\subsection{Złożoność obliczeniowa}
Dla drzewa złożoność czasowa DFS'a wynosi $\mathcal{O}(n)$. Być może drobnego wyjaśnienia wymaga czas sprawdzenia warunku $d+d(v,u) \leq C$. Zauważmy, że iterując się po liście sąsiadów $v$ jesteśmy obecnie w elemencie odpowiadającym wierzchołkowi $u$, zatem dostęp do wagi związanej z krawędzią $(v, u)$ jest w czasie stałym (waga ta jest wpisana w tym elemencie listy sąsiadów). Zastanówmy się więc, jaka jest złożoność procedury Algorytm. Oznaczenie wszystkich wierzchołków jako nieodwiedzonych wymaga każdorazowo $\mathcal{O}(n)$ operacji. DFS tak samo. Pętla, w której znajdują się te instrukcje wykonuje się tyle razy, ile jest wierzchołków w naszym grafie, czyli $n$. Zatem wykonanie się wszystkich obrotów pętli wymaga $\mathcal{O}(n^2)$ czasu. Instrukcje wykonywane przed i po pętli wykonują się w czasie stałym. Stąd ogólna złożoność obliczeniowa algorytmu wynosi $\mathcal{O}(n^2)$.

\end{document}