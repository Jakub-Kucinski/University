\documentclass{article}

\usepackage{polski}
\usepackage[utf8]{inputenc}

\usepackage{fancyhdr} % Required for custom headers
\usepackage{lastpage} % Required to determine the last page for the footer
\usepackage{extramarks} % Required for headers and footers
\usepackage[usenames,dvipsnames]{color} % Required for custom colors
\usepackage{graphicx} % Required to insert images
\usepackage{listings} % Required for insertion of code
\usepackage{courier} % Required for the courier font
\usepackage{lipsum} 
\usepackage{amsfonts}
\usepackage{hyperref}
\usepackage{tikz}
\usepackage{amsmath}
\usepackage{pdfpages}
\usepackage{mathtools}
\usepackage{float}
\usepackage{amsthm}
 
\usepackage[noend]{algpseudocode}
\usepackage{algorithm}
\usepackage{algorithmicx}

\newtheorem{thm}{Twierdzenie}
\newtheorem{remark}{Uwaga}
\newtheorem{lemat}{Lemat}
\newtheorem{wniosek}{Wniosek}
\newtheorem{definicja}{Definicja}
\newtheorem{ciekawostka}{Ciekawostka}
\newtheorem{przyklad}{Przykład}

\newenvironment{rozw}{\paragraph{Rozwiązanie:}}{\hfill}


% Margins
\topmargin=-0.45in
\evensidemargin=0in
\oddsidemargin=0in
\textwidth=6.6in
\textheight=9.0in
\headsep=0.25in

\linespread{1.1} % Line spacing


\title{Zadanie 5 lista 6}
\author{Jakub Kuciński, prowadzący Szymon Dudycz}

\begin{document}

\maketitle

\tableofcontents

\section{Rozwiązanie}
Chcemy znaleźć k-ty największy element. Pivot jest wybierany losowo spośród elementów tablicy. Chcemy oszacować oczekiwany czas działania algorytmu. Czas działania algorytmu jest asymptotycznie równy liczbie porównań elementów tablicy. Niech $y_1, y_2, \ldots, y_n$ będą elementami tablicy w kolejności rosnącej. Zauważmy, że jedynym momentem kiedy elementy są porównywane jest moment, gdy jeden z nich jest pivotem. Po stworzeniu zbiorów elementów większych i mniejszych od pivota możemy sprawdzić, czy pivot jest k-ty co do wielkości, dzięki czemu nie będzie brał udział w dalszym wyszukiwaniu k-tego elementu, więc też nigdy więcej nie zostanie porównany. Wiemy zatem, że każde dwa elementy zostaną porównane co najwyżej raz. Wprowadźmy zmienne losowe 
$$
X_{i, j} =
\begin{cases}
    1, & \text{jeśli $y_i$ oraz $y_j$ zostają porównane przez algorytm} \\
    0, & \text{w przeciwnym przypadku}
\end{cases}
$$
Zatem oczekiwany czas działania algorytmu to oczekiwana wartość sumy zmiennych losowych $X_{i, j}$.
\begin{equation*}
    E\left[\sum_{i,j} X_{i, j}\right] = \sum_{i,j} E\left[ X_{i, j}\right]
\end{equation*}
Ustalmy k. Rozważmy przypadki:
\begin{enumerate}
    \item $i<j\leq k$\\
    Jeśli pivotem zostanie wybrany któryś z elementów $y_{k+1}, \ldots, y_n$ to algorytm nie odrzuci ani $y_i$ ani $y_j$. Czyli wybór takiego pivota nie wpływa na szansę porównania elementów $y_i$ i $y_j$. Podobnie wybór pivota spośród elementów $y_1, \ldots, y_{i-1}$ nie wpływa na tą szansę. Jeśli jednak wybrany na pivota zostanie któryś z elementów $y_i, \ldots, y_k$ to któryś (być może oba) z elementów $y_i$ i $y_j$ zostanie odrzucony, więc nie będzie mógł zostać już porównany z drugim. Czyli jedyną szansą na porównanie $y_i$ i $y_j$ jest wybór jednego z nich na pivota spośród elementów $y_i, \ldots, y_k$.\\
    Możemy policzyć sumę wartości oczekiwanych elementów spełniających powyższy warunek.
    \begin{align*}
        &\sum_{i=1}^{k-1}\sum_{j=i+1}^k \frac{2}{k-i+1} = \sum_{i=1}^{k-1} (k-i) \frac{2}{k-i+1} = \sum_{i=1}^{k-1} i \frac{2}{i+1} \leq \sum_{i=1}^k 2 = 2k = \mathcal{O}(n)
    \end{align*}
    \item $k\leq i < j$\\
        Przypadek analogiczny do poprzedniego. Możemy zauważyć, że pytanie o k-ty najmniejszy jest równoważny pytaniu o n-k największy. Wtedy dostaniemy ograniczenie z góry zamiast $2k$ jak w poprzednim przypadku to $2(n-k)$ co jest rzędu $\mathcal{O}(n)$.
    \item $i<k<j$
        W tym przypadku wybór pivota spośród $y_1, \ldots, y_{i-1}$ lub $y_{j+1}, \ldots, y_{n}$ nie wpływa na szansę porównania $y_i$ i $y_j$. Jeśli jednak wybrany na pivota zostanie któryś z elementów $y_i, \ldots, y_j$ to któryś (być może oba) z elementów $y_i$ i $y_j$ zostanie odrzucony, więc nie będzie mógł zostać już porównany z drugim. Czyli jedyną szansą na porównanie $y_i$ i $y_j$ jest wybór jednego z nich na pivota spośród elementów $y_i, \ldots, y_j$.\\
    \begin{align*}
        &\sum_{i=1}^{k-1}\sum_{j=k+1}^n \frac{2}{j-i+1} = \sum_{i=1}^{k-1} (k-i) \frac{2}{k-i+1} = \sum_{i=1}^{k-1} \sum_{j=k-i+1}^{n-i} \frac{2}{j+1} = 2\sum_{i=1}^{k-1} (H_{n-i+1} - H_{k-i+1}) \leq\\
        &\leq 2\sum_{i=1}^{k-1} (\ln{(n-i+1)} + 1 - \ln{(k-i+2)}) \leq 2\sum_{i=1}^{k-1} (\ln{(n-i+1)} + 1 - \ln{(k-i)}) = 2\sum_{i=1}^{k-1} (\ln{(\frac{n-i+1}{k-i})} + 1) = \\
        & = 2k + 2\ln{( \frac{n!}{(k-1)! (n-k+1)!} )} = 2k + 2\ln{({n \choose k-1})} \leq 2k + 2\ln{(2^n)} = 2k + 2n\ln{(2)} = \mathcal{O}(n)
    \end{align*}
\end{enumerate}
Suma wszystkich wartości oczekiwanych jest sumą omówionych powyżej przypadków, a skoro wszystkie są rzędu $\mathcal{O}(n)$ to ich suma też jest rzędu $\mathcal{O}(n)$. Zatem oczekiwana złożoność naszego algorytmu wynosi $\mathcal{O}(n)$.
\end{document}