\documentclass{article}

\usepackage{polski}
\usepackage[utf8]{inputenc}

\usepackage{fancyhdr} % Required for custom headers
\usepackage{lastpage} % Required to determine the last page for the footer
\usepackage{extramarks} % Required for headers and footers
\usepackage[usenames,dvipsnames]{color} % Required for custom colors
\usepackage{graphicx} % Required to insert images
\usepackage{listings} % Required for insertion of code
\usepackage{courier} % Required for the courier font
\usepackage{lipsum} 
\usepackage{amsfonts}
\usepackage{hyperref}
\usepackage{tikz}
\usepackage{amsmath}
\usepackage{pdfpages}
\usepackage{mathtools}
\usepackage{float}
\usepackage{amsthm}
 
\usepackage[noend]{algpseudocode}
\usepackage{algorithm}
\usepackage{algorithmicx}

\newtheorem{thm}{Twierdzenie}
\newtheorem{remark}{Uwaga}
\newtheorem{lemat}{Lemat}
\newtheorem{wniosek}{Wniosek}
\newtheorem{definicja}{Definicja}
\newtheorem{ciekawostka}{Ciekawostka}
\newtheorem{przyklad}{Przykład}

\newenvironment{rozw}{\paragraph{Rozwiązanie:}}{\hfill}


% Margins
\topmargin=-0.45in
\evensidemargin=0in
\oddsidemargin=0in
\textwidth=6.6in
\textheight=9.0in
\headsep=0.25in

\linespread{1.1} % Line spacing


\title{Zadanie 2 lista 4}
\author{Jakub Kuciński, prowadzący Szymon Dudycz}

\begin{document}

\maketitle

\tableofcontents

\section{Treść zadania}
Zbiór $I\subseteq V$ zbioru wierzchołków w grafie $G=(V, E)$ nazywamy zbiorem niezależnym, jeśli żadne dwa wierzchołki z $I$ nie są połączone krawędzią. Ułóż algorytm, który dla zadanego drzewa T znajduje najliczniejszy zbiór niezależny jego wierzchołków.

\section{Idea rozwiązania}
Ukorzeńmy graf w dowolnym wierzchołku. Dla dowolnego wierzchołka $v$ zastanówmy się jaki jest maksymalny zbiór niezależny $I_v$ w tworzonym przez niego poddrzewie. Wierzchołek $v$ może należeć lub nie należeć do maksymalnego zbioru niezależnego, stąd mamy przypadki:
\begin{enumerate}
    \item $v \in I_v$ \\
    Wtedy żadne $u$ dziecko $v$ nie może należeć do $I_v$. Wystarczy zatem znaleźć najmniejsze zbiory niezależne poddrzew o korzeniach w $u$, do których nie należą $u$. Zatem $I_v$ będzie ich sumą mnogościową powiększoną o $v$.
    \item $v \notin I_v$ \\
    Wtedy zbiory niezależne poddrzew utworzonych w $u$ dzieciach $v$ mogą dowolnie zawierać lub nie zawierać $u$. Czyli $I_v$ będzie sumą mnogościową większych ze zbiorów niezależnych zawierających i niezawierających $u$.
\end{enumerate}
Z powyższych rozważań widzimy, że wystarczy dla każdego wierzchołka policzyć maksymalne zbiory niezależne zawierające i niezawierające ich dzieci dla poddrzew o korzeniach w dzieciach. Oczywiście dla liścia $w$ mamy odpowiednio $\{ w \}$ i $\emptyset$. Aby szybko rozpoznać który zbiór niezależny jest większy wystarczy oprócz jego elementów przechowywać również jego rozmiar.

\section{Algorytm}
\begin{algorithm}
\caption{$Zbior\_niezalezny(v)$}
\begin{algorithmic}[1]
\State $S_{with} \leftarrow \{ v \}$
\State $S_{without} \leftarrow \emptyset$
\For{$u \in N(v)$}
    \If{$u$ nieodwiedzony}
        \State oznacz $u$ jako odwiedzony
        \State $A_{with}, A_{without} \leftarrow Zbior\_niezalezny(u)$
        \State $S_{with}. \leftarrow S_{with} \cup A_{without}$
        \If{$A_{with}.size > A_{without}.size$}
            \State $S_{without} \leftarrow S_{without} \cup A_{with} $
        \Else
            \State $S_{without} \leftarrow S_{without} \cup A_{without} $
        \EndIf
    \EndIf
\EndFor\\
\Return $S_{with}, S_{without}$
\end{algorithmic}
\end{algorithm}
W celu obliczenia największego zbioru niezależnego drzewa wystarczy wywołać powyższą procedurę na dowolnym początkowym wierzchołku i zwrócić większy z otrzymanych zbiorów.

\section{Złożoność obliczeniowa i pamięciowa}
\subsection{Złożoność pamięciowa}
Utworzenie listy sąsiedztwa wymaga $\mathcal{O}(n)$ pamięci. Zbiory konstruowane przez nasz algorytm mogą być zwykłymi listami wiązanymi, dla których trzymamy wskaźniki na pierwszy i ostatni element oraz rozmiar zbioru. Wówczas łączenie zbiorów polega na podpięciu pewnej listy wiązanej na koniec innej listy wiązanej oraz zmiany rozmiaru zbioru na sumę rozmiarów łączonych zbiorów. W końcowym zbiorze może znaleźć się co najwyżej $n-1$ wierzchołków, stąd zbiory wymagają $\mathcal{O}(n)$ pamięci. Informacje, czy dany wierzchołek został odwiedzony można trzymać w dodatkowej tablicy. Wymaga to $\mathcal{O}(n)$ pamięci. Ostatecznie algorytm zużywa $\mathcal{O}(n)$ pamięci.

\subsection{Złożoność obliczeniowa}
Każdy wierzchołek zostanie odwiedzony dokładnie raz, tak jak przy zwykłym DFS'ie. Operacje inicjalizacji, sumowania zbiorów, aktualizacji i sprawdzania ich rozmiarów wykonują się w czasie stałym. Stąd złożoność czasowa wynosi $\mathcal{O}(n)$.

\end{document}