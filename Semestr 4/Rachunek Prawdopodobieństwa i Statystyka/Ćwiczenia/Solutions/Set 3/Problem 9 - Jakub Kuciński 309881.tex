\documentclass[11pt,wide]{article}


\usepackage[utf8]{inputenc}

\usepackage{graphicx} 

\usepackage{mathtools}
\usepackage{amsthm}
\usepackage{amsmath}
\usepackage{verbatim}
\usepackage{xcolor}

\usepackage{hyperref}
\usepackage{ textcomp }

\hypersetup{
    colorlinks=true,
    linkcolor=blue,
    filecolor=magenta,      
    urlcolor=cyan,
    citecolor=green,
    pdftitle={Sharelatex Example},
    bookmarks=true,
}

\newcommand\numeq[1]%
  {\stackrel{\scriptscriptstyle \mkern-1.5mu#1\mkern-1.5mu }{=}}

\newtheorem{thm}{Twierdzenie}
\newtheorem{remark}{Uwaga}
\newtheorem{lemat}{Lemat}
\newtheorem{wniosek}{Wniosek}
\newtheorem{definicja}{Definicja}
\newtheorem{ciekawostka}{Ciekawostka}
\newtheorem{przyklad}{Przykład}
\newtheorem{rysunek}{Rysunek}

% Marginesy
\topmargin=-0.45in
\evensidemargin=0in
\oddsidemargin=0in
\textwidth=6.5in
\textheight=9.0in
\headsep=0.25in

\title{Problem set 3\textdegree}
\date{Wrocław, March 14, 2020}
\author{Jakub Kuciński, Pratik Ghosal}

\begin{document}

\maketitle
\thispagestyle{empty} 
\tableofcontents


\section{Problem 9\textdegree}
\subsection{a)}
As $X$ has uniform distribution and $x \in [-2,2]$ then we have $f_X(x)=\frac{1}{4}$ and consequently \\
\(\displaystyle F_X(s) = \int_{-2}^s \frac{1}{4} \mathop{dx} = \frac{s+2}{4}\).\\
\(Y = |X|\)\\
$t \in [0,2]$\\
\(F_Y(t) = P(Y<t) = P(|X|<t) = P(-t<X<t)\)\\
\(F_Y(t) = P(X<t) - P(X<-t) = F_X(t) - F_X(-t)\).\\
\(\displaystyle F_Y(t) = F_X(t) - F_X(-t) = \int_{-2}^t \frac{1}{4} \mathop{dx} - \int_{-2}^{-t} \frac{1}{4} \mathop{dx} = 
\frac{t+2}{4} - \frac{-t+2}{4} = \frac{t}{2} \).\\
As \(\displaystyle \left( F_Y(t)\right) ' = f_Y(t) \) we get \(\displaystyle f_Y(t) = \left( \frac{t}{2}\right) ' = \frac{1}{2} \).

\subsection{b)}
As $X$ has uniform distribution and $x \in [-1,1]$ then we have 
$f_X(x)=\frac{1}{2}$ 
and consequently \\
\(\displaystyle F_X(s) = \int_{-1}^s \frac{1}{2} \mathop{dx} = \frac{s+1}{2}\).\\
\(Y = X^3\)\\
\( t\in [-1,1]\)\\
\(F_Y(t) = P(Y<t) = P(X^3<t) = P(X<\sqrt[3]{t}) = F_X(\sqrt[3]{t}) \) \\
\(\displaystyle (F_Y(t))' = (F_X(\sqrt[3]{t}))' \) \\
\(\displaystyle f_Y(t) = f_X(\sqrt[3]{t}) \frac{1}{3\sqrt[3]{t^2}} = \frac{1}{2} \cdot \frac{1}{3\sqrt[3]{t^2}} = 
\frac{1}{6\sqrt[3]{t^2}} \) \\ \\
\(Z = X^2\)\\
\( t\in [0,1]\)\\
\(\displaystyle F_Z(t) = P(Z<t) = P(X^2<t) = P(-\sqrt{t}<X<\sqrt{t}) = P(X<\sqrt{t}) - P(X<-\sqrt{t}) = \) \\
\(\displaystyle = F_X(\sqrt{t}) - F_X(-\sqrt{t}) \) \\
\(\displaystyle F_Z(t) = F_X(\sqrt{t}) - F_X(-\sqrt{t}) \) \\
\(\displaystyle (F_Z(t))' = (F_X(\sqrt{t}) - F_X(-\sqrt{t}))' \) \\
\(\displaystyle f_Z(t) = f_X(\sqrt{t}) \cdot \frac{1}{2\sqrt{t}} + f_X(-\sqrt{t}) \cdot \frac{1}{2\sqrt{t}} = \frac{1}{2\sqrt{t}} \)

\end{document}