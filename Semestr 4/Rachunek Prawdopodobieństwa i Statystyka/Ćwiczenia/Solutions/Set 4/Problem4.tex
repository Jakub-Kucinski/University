\documentclass[11pt,wide]{article}


\usepackage[utf8]{inputenc}

\usepackage{graphicx} 

\usepackage{mathtools}
\usepackage{amsthm}
\usepackage{amsmath}
\usepackage{verbatim}
\usepackage{xcolor}

\usepackage{hyperref}
\usepackage{ textcomp }


\hypersetup{
    colorlinks=true,
    linkcolor=blue,
    filecolor=magenta,      
    urlcolor=cyan,
    citecolor=green,
    pdftitle={Sharelatex Example},
    bookmarks=true,
}

\newcommand\numeq[1]%
  {\stackrel{\scriptscriptstyle \mkern-1.5mu#1\mkern-1.5mu }{=}}

\newtheorem{thm}{Twierdzenie}
\newtheorem{remark}{Uwaga}
\newtheorem{lemat}{Lemat}
\newtheorem{wniosek}{Wniosek}
\newtheorem{definicja}{Definicja}
\newtheorem{ciekawostka}{Ciekawostka}
\newtheorem{przyklad}{Przykład}
\newtheorem{rysunek}{Rysunek}

% Marginesy
\topmargin=-0.45in
\evensidemargin=0in
\oddsidemargin=0in
\textwidth=6.5in
\textheight=9.0in
\headsep=0.25in

\title{Problem 4\textdegree}

\begin{document}

\maketitle
\thispagestyle{empty} 

Proof that \(\displaystyle \Gamma\left(\frac{1}{2}\right) = \sqrt{\pi}\). \\
\begin{center}
\(\displaystyle \Gamma\left(\frac{1}{2}\right) = \int_0^{\infty} t^{-\frac{1}{2}}e^{-t}\mathop{dt} = \int_0^{\infty} \frac{1}{\sqrt{t}}e^{-t} \mathop{dt} = \bigg| t=\frac{x^2}{2}, \mathop{dt} = x\mathop{dx} \bigg| = \int_0^{\infty} \frac{\sqrt{2}}{x} e^{-x^2/2} x \mathop{dx} = \sqrt{2} \int_0^{\infty} e^{-x^2/2} \mathop{dx} = \sqrt{2} \cdot \frac{1}{2} \cdot \sqrt{2\pi}= \sqrt{\pi}\) \\
\end{center}

Equation \(\displaystyle \int_0^{\infty} e^{-x^2/2} \mathop{dx} = \frac{1}{2} \cdot \sqrt{2\pi} \) holds as from 1.6 we know \(\displaystyle \int_{-\infty}^{\infty} e^{-x^2/2} \mathop{dx} = \sqrt{2\pi} \) \\
and from evenness of \(\displaystyle e^{-x^2/2} \) we have \(\displaystyle \int_{-\infty}^{\infty} e^{-x^2/2}\mathop{dx} = 2 \int_0^{\infty} e^{-x^2/2} \mathop{dx} \).

\end{document}