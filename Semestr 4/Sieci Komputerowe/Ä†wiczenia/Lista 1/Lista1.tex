\documentclass[11pt,wide]{article}

\usepackage{polski}
\usepackage[utf8]{inputenc}

\usepackage{graphicx} 

\usepackage{mathtools}
\usepackage{amsthm}
\usepackage{verbatim}
\usepackage{xcolor}

\usepackage{hyperref}


\hypersetup{
    colorlinks=true,
    linkcolor=blue,
    filecolor=magenta,      
    urlcolor=cyan,
    citecolor=green,
    pdftitle={Sharelatex Example},
    bookmarks=true,
}

\newcommand\numeq[1]%
  {\stackrel{\scriptscriptstyle \mkern-1.5mu#1\mkern-1.5mu }{=}}

\newtheorem{thm}{Twierdzenie}
\newtheorem{remark}{Uwaga}
\newtheorem{lemat}{Lemat}
\newtheorem{wniosek}{Wniosek}
\newtheorem{definicja}{Definicja}
\newtheorem{ciekawostka}{Ciekawostka}
\newtheorem{przyklad}{Przykład}
\newtheorem{rysunek}{Rysunek}

% Marginesy
\topmargin=-0.45in
\evensidemargin=0in
\oddsidemargin=0in
\textwidth=6.5in
\textheight=9.0in
\headsep=0.25in

\title{Sieci Komputerowe - Lista 1}
\date{Wrocław, Marzec 19, 2020}
\author{Jakub Kuciński, prowadzący Andrzej Łukaszewski}

\begin{document}

\maketitle
\thispagestyle{empty} 
\tableofcontents


\section{Zadanie 1}
Kropki w numerze IP rozdzielają bajty danego adresu IP. Adres rozgłoszeniowy jest ostatnim adresem, adres sieci to pierwszy adres, a adresy komputerów to wszystkie pomiędzy nimi. Liczba po / oznacza długość prefiksu sieci (w bitach).
\begin{itemize}
\item 10.1.2.3/8  : podany adres to adres komputera, adres sieci - 10.0.0.0, adres rozgłoszeniowy 10.255.255.255, adres jakiegoś komputera - 10.0.0.1
\item 156.17.0.0/16 : podany adres to adres sieci, adres sieci - 156.17.0.0/16, adres rozgłoszeniowy - 156.17.255.255, adres jakiegoś komputera - 156.17.11.11
\item 99.99.99.99/27 = 01100011.01100011.01100011.01100011/27 : podany adres to adres komputera, adres sieci - 99.99.99.96, adres rozgłoszeniowy - 99.99.99.127, adres jakiegoś komputera - 99.99.99.100
\item 156.17.64.4/30 : podany adres to adres sieci, adres sieci - 156.17.64.4, adres rozgłoszeniowy - 156.17.64.7, adres jakiegoś komputera - 156.17.64.5
\item 123.123.123.123/32 : jest to sieć z jednym adresem IP (jeden komputer), w tym przypadku adres rozgłoszeniowy, sieci i komputera są podanym adresem IP
\end{itemize}

\section{Zadanie 2}
Adres sieci 10.10.0.0/16 z zadania zapisany binarnie: 00001010.00001010.00000000.00000000 i jej maska 11111111.11111111.00000000.00000000. Możemy podzielić tę sieć na pół. Pierwszą połowę przeznaczymy na pierwszą sieć, a drugą podzielimy na kolejno 4 równoliczne rozłączne sieci:\\
\begin{itemize}
\item Adres pierwszej (większej) sieci - 00001010.00001010.00000000.00000000, \\maska - 11111111.11111111.10000000.00000000. Zamieniając na zapis dziesiętny - 10.10.0.0/17
\item Adres drugiej sieci - 00001010.00001010.10000000.00000000, \\maska - 11111111.11111111.11100000.00000000. Zamieniając na zapis dziesiętny - 10.10.128.0/19
\item Adres trzeciej sieci -- 00001010.00001010.10100000.00000000, \\maska - 11111111.11111111.11100000.00000000. Zamieniając na zapis dziesiętny - 10.10.160.0/19
\item Adres czwartej sieci -- 00001010.00001010.11000000.00000000, \\maska - 11111111.11111111.11100000.00000000. Zamieniając na zapis dziesiętny - 10.10.192.0/19
\item Adres piątej sieci - - 00001010.00001010.10100000.00000000, \\maska - 11111111.11111111.11100000.00000000. Zamieniając na zapis dziesiętny - 10.10.224.0/19
\end{itemize}
Każda z tych podsieci zawiera dwa adresy, które nie mogą być przypisane komputerom - adres sieci i adres rozgłoszeniowy. Skoro mamy 5 podsieci to łącznie 10 adresów nie może zostać wykorzystanych do adresowania komputerów. Natomiast oryginalna sieć 10.10.0.0/16 zawierała już adres rozgłoszeniowy i adres sieci, zatem liczba niemożliwych do przypisania adresów wzrosła o 8.\\\\
Jaka jest najmniejsza możliwa sieć? Powiedzmy, że ma prefiks długości k. Nasze 5 podsieci jest rozłącznych i wypełniają całą sieć 10.10.0.0/16, zatem musi istnieć inna podsieć, która ma prefiks takiej samej długości jak k, bo musi ją "uzupełnić" do większej sieci. Razem tworzą sieć o prefiksie długości k-1. Ona z kolei znowu musi być uzupełniona przez inną sieć o prefiksie k-1. Razem tworzą sieć o prefiksie k-2. Znowu uzupełniamy, łączymy i uzupełniamy. Mamy sieć o prefiksie k-4 i wykorzystaliśmy do tego już wszystkie dostępne 5 podsieci, zatem musieliśmy dostać sieć 10.10.0.0/16, czyli $k-4=16 $, stąd $k = 20$. Zatem najmniejsza sieć ma $2^{32-20} - 2 = 4094$ adresów dostępnych dla komputerów.
\section{Zadanie 3}
\begin{itemize}
\item[1)] 0.0.0.0/0 $\rightarrow$ do routera A
\item[2)] 10.0.0.0/23 $\rightarrow$ do routera B
\item[3)] 10.0.2.0/24 $\rightarrow$ do routera B
\item[4)] 10.0.3.0/24 $\rightarrow$ do routera B
\item[5)] 10.0.1.0/24 $\rightarrow$ do routera C
\item[6)] 10.0.0.128/25 $\rightarrow$ do routera B
\item[7)] 10.0.1.8/29 $\rightarrow$ do routera B
\item[8)] 10.0.1.16/29 $\rightarrow$ do routera B
\item[9)] 10.0.1.24/29 $\rightarrow$ do routera B
\end{itemize}
Wiemy, że w tablicy routingu wybierana jest reguła z najdłuższym prefiksem.
Jeśli miniemy wpis 5), to możemy wpaść do routera B albo do ogólnego A. Zauważmy, że wtedy możemy zapisać 2), 3) i 4) łącznie jako 10.0.0.0/22 $\rightarrow$ do routera B. Widzimy, że wpisy 5)-9) są uszczegółowieniami 10.0.0.0/22 oraz występują w nich tylko routery B i C. Możemy zauważyć, że wpis 6) jest rozłączny z jedynym wpisem z routerem C, czyli 5), więc skoro uszczegóławia 10.0.0.0/22 $\rightarrow$ do routera B to możemy go pominąć (bo też prowadzi do B). Wpisy 7)-9) tworzą zwartą część, która uszczegóławia wpis 5). Możemy zatem zastąpić je dwoma wpisami. Mniej szczegółowym 10.0.1.0/27 $\rightarrow$ do routera B oraz bardziej szczegółowym 10.0.1.0/29 $\rightarrow$ do routera C. W ten sposób otrzymujemy tablicę routingu:
\begin{itemize}
\item 0.0.0.0/0 $\rightarrow$ do routera A
\item 10.0.0.0/22 $\rightarrow$ do routera B
\item 10.0.1.0/24 $\rightarrow$ do routera C
\item 10.0.1.0/27 $\rightarrow$ do routera B
\item 10.0.1.0/29 $\rightarrow$ do routera C
\end{itemize}

\section{Zadanie 4}
\begin{itemize}
\item[1)] 0.0.0.0/0 $\rightarrow$ do routera A
\item[2)] 10.0.0.0/8 $\rightarrow$ do routera B
\item[3)] 10.3.0.0/24 $\rightarrow$ do routera C
\item[4)] 10.3.0.32/27 $\rightarrow$ do routera B
\item[5)] 10.3.0.64/27 $\rightarrow$ do routera B
\item[6)] 10.3.0.96/27 $\rightarrow$ do routera B
\end{itemize}
Analogicznie jak w zadaniu 3. Widzimy, że wpisy 3)-6) są uszczegółowieniami 10.0.0.0/8 oraz występują w nich tylko routery B i C. Wpisy 4)-6) tworzą zwartą część, zawartą we wpisie 3). Nieprzysłonięty przez wpisy 4)-6) fragment wpisu 3) można rozbić na dwie części 10.3.0.128/25 oraz 10.3.0.0/27. Nie będą one miały żadnego wspólnego adresu z wpisami 4)-6), a skoro 4)-6) leżą wewnątrz 2) to będzie można je pominąć. W ten sposób otrzymamy:
\begin{itemize}
\item 0.0.0.0/0 $\rightarrow$ do routera A
\item 10.0.0.0/8 $\rightarrow$ do routera B
\item 10.3.0.128/25 $\rightarrow$ do routera C
\item 10.3.0.0/27 $\rightarrow$ do routera C
\end{itemize}


\section{Zadanie 5}
Teza: uporządkowanie wpisów w tablicy routingu według długości prefiksu od najdłuższego do najkrótszego i wybór pierwszego pasującego jest równoważne braniu najlepszego dopasowania (najdłuższy pasujący prefiks).\\\\
Dowód: \\
Załóżmy nie wprost, że wybór pierwszego pasującego z naszej uporządkowanej tablicy routingu nie jest równoważny braniu najlepszego dopasowania. Weźmy adres ip, dla którego teza nie zachodzi i nazwijmy go Addr. Niech WpisUporz oznacza pierwszy wpis pasujący do Addr w uporządkowanej tablicy oraz WpisNajl oznacza wpis, który jest najlepszym dopasowaniem (najdłuższy prefiks). Ponadto przez PreUporz oznaczmy długość prefiksu WpisUporz oraz przez PreNajl oznaczmy długość prefiksu WpisNajl. Rozważmy przypadki:
\begin{itemize}
\item PreUporz $>$ PreNajl \\
Długość prefisku WpisNajl jest z założenia największa. Ale założyliśmy, że PreUporz $>$ PreNajl. Sprzeczność.
\item PreUporz = PreNajl \\
Wiemy, że PreNajl to największa długość prefiksu wpisu pasującego do Addr, a skoro PreUporz = PreNajl, to WpisUporz też jest najbardziej pasującym wpisem. Sprzeczność (bo PreUporz miał być nienajlepszy).
\item PreUporz $<$ PreNajl \\
Wiemy, że WpisNajl znajduje się w tablicy. Zastanówmy się gdzie znajduje sie w tablicy uporządkowanej. Skoro ma dłuższy prefiks to znajduje się przed WpisUporz, ale skoro pasuje do Addr i znajduje się przed WpisUporz, a bierzemy pierwszy pasujący wpis, to powinniśmy wziąć WpisNajl, a nie WpisUporz. Sprzeczność.
\end{itemize}
Z powyższych przypadków dochodzimy do sprzeczności, a stąd teza była prawdziwa.

\section{Zadanie 6}
Początkowy stan tablicy routingu:\\
\begin{tabular}{|c|c|c|c|c|c|c|}\hline
 & A & B & C & D & E & F\\
\hline 
do A & - & 1 & & & &\\
\hline 
do B & 1 & - & 1 & & &\\
\hline 
do C & & 1 & - & & 1 & 1\\
\hline 
do D & & & & - & 1 &\\
\hline 
do E & & & 1 & 1 & - & 1\\
\hline
do F & & & 1 & & 1 & - \\
\hline
do S & 1 & 1 & & & & \\
\hline
\end{tabular}\\\\

\begin{tabular}{|c|c|c|c|c|c|c|}\hline
 	 & A & B & C 		 & D & E & F\\
\hline 
do A & - & 1 & 2 (via B) & & &\\
\hline 
do B & 1 & - & 1 		 & & 2 (via C) & 2 (via C)\\
\hline 
do C & 2 (via B) & 1 & - & 2 (via E) & 1 & 1\\
\hline 
do D & & & 2 (via E) & - & 1 & 2 (via E)\\
\hline 
do E & & 2 (via C) & 1 & 1 & - & 1\\
\hline
do F & & 2 (via C) & 1 & 2 (via E) & 1 & -\\
\hline
do S & 1 & 1 & 2 (via B) & & & \\
\hline
\end{tabular}\\\\

\begin{tabular}{|c|c|c|c|c|c|c|}\hline
 & A & B & C & D & E & F\\
\hline 
do A & - & 1 & 2 (via B) &  & 3 (via C) &  3 (via C)\\
\hline 
do B & 1 & - & 1 & 3 (via E) & 2 (via C) & 2 (via C)\\
\hline 
do C & 2 (via B) & 1 & - & 2 (via E) & 1 & 1\\
\hline 
do D & 			 & 3 (via C) & 2 (via E) & - & 1 & 2 (via E)\\
\hline 
do E & 3 (via B) & 2 (via C) & 1 & 1 & - & 1\\
\hline
do F & 3 (via B) & 2 (via C) & 1 & 2 (via E) & 1 & -\\
\hline
do S & 1 & 1 & 2 (via B) & & 3 (via C) & 3 (via C) \\
\hline
\end{tabular}\\\\

\begin{tabular}{|c|c|c|c|c|c|c|}\hline
 & A & B & C & D & E & F\\
\hline 
do A & - & 1 & 2 (via B) & 4 (via E)  & 3 (via C) &  3 (via C)\\
\hline 
do B & 1 & - & 1 & 3 (via E) & 2 (via C) & 2 (via C)\\
\hline 
do C & 2 (via B) & 1 & - & 2 (via E) & 1 & 1\\
\hline 
do D & 4 (via B) & 3 (via C) & 2 (via E) & - & 1 & 2 (via E)\\
\hline 
do E & 3 (via B) & 2 (via C) & 1 & 1 & - & 1\\
\hline
do F & 3 (via B) & 2 (via C) & 1 & 2 (via E) & 1 & -\\
\hline
do S & 1 & 1 & 2 (via B) & 4 (via E) & 3 (via C) & 3 (via C) \\
\hline
\end{tabular}\\\\
Jak widać stan stabilny zostanie osiągnięty po 3 krokach

\section{Zadanie 7}
Tablica routingu po dodaniu połączenia między routerami A i D. \\
\begin{tabular}{|c|c|c|c|c|c|c|}\hline
 & A & B & C & D & E & F\\
\hline 
do A & - & 1 & 2 (via B) & 1  & 3 (via C) &  3 (via C)\\
\hline 
do B & 1 & - & 1 & 3 (via E) & 2 (via C) & 2 (via C)\\
\hline 
do C & 2 (via B) & 1 & - & 2 (via E) & 1 & 1\\
\hline 
do D & 1 & 3 (via C) & 2 (via E) & - & 1 & 2 (via E)\\
\hline 
do E & 3 (via B) & 2 (via C) & 1 & 1 & - & 1\\
\hline
do F & 3 (via B) & 2 (via C) & 1 & 2 (via E) & 1 & -\\
\hline
do S & 1 & 1 & 2 (via B) & 4 (via E) & 3 (via C) & 3 (via C) \\
\hline
\end{tabular}\\\\

Uaktualniona tablica routingu: \\
\begin{tabular}{|c|c|c|c|c|c|c|}\hline
 & A & B & C & D & E & F\\
\hline 
do A & - & 1 & 2 (via B) & 1 & 2 (via D) &  3 (via C)\\
\hline 
do B & 1 & - & 1 & 2 (via A) & 2 (via C) & 2 (via C)\\
\hline 
do C & 2 (via B) & 1 & - & 2 (via E) & 1 & 1\\
\hline 
do D & 1 & 2 (via A) & 2 (via E) & - & 1 & 2 (via E)\\
\hline 
do E & 2 (via D) & 2 (via C) & 1 & 1 & - & 1\\
\hline
do F & 3 (via B) & 2 (via C) & 1 & 2 (via E) & 1 & -\\
\hline
do S & 1 & 1 & 2 (via B) & 2 (via A) & 3 (via C) & 3 (via C) \\
\hline
\end{tabular}


\section{Zadanie 8}
Trasa do E po awarii łącza między D a E (bez straty ogólności przymujemy, że ścieżka z A do E prowadziła przez B a nie C):\\
\begin{tabular}{|c|c|c|c|c|}\hline
 & A & B & C & D \\
 \hline
 do E & 3 (via B) & 2 (via D) & 2 (via D) & $\infty$\\
 \hline
\end{tabular}\\

D wysyła informacje do B i C o swoim sąsiedztwie. Interesuje nas to, że jego ścieżka do E wynosi $\infty$.
Skoro droga z B i C prowadziła do E przez D to nowe drogi wynoszą teraz $\infty$.\\
\begin{tabular}{|c|c|c|c|c|}\hline
 & A & B & C & D \\
 \hline
 do E & 3 (via B) & $\infty$ & $\infty$ & $\infty$\\
 \hline
\end{tabular}\\

Następnie A wysyła do B zatrutą ścieżkę (bo droga do E prowadzi przez B) oraz informuje C, że ma ścieżkę długości 3 do E. Droga z C do E przez A jest krótsza od $\infty$, więc zostaje wpisana do C.\\
\begin{tabular}{|c|c|c|c|c|}\hline
 & A & B & C & D \\
 \hline
 do E & 3 (via B) & $\infty$ & 4 (via A) & $\infty$\\
 \hline
\end{tabular}\\

C wysyła do A zatrutą ścieżkę oraz informuje D, że ma ścieżkę do E przez A. D uaktualnia swoją ścieżkę do E.\\
\begin{tabular}{|c|c|c|c|c|}\hline
 & A & B & C & D \\
 \hline
 do E & 3 (via B) & $\infty$ & 4 (via A) & 5 (via C)\\
 \hline
\end{tabular}\\

Teraz D wysyła informacje do B, że ma ścieżkę do E przez C. B uaktualnia swoją ścieżkę do E.\\
\begin{tabular}{|c|c|c|c|c|}\hline
 & A & B & C & D \\
 \hline
 do E & 3 (via B) & 6 (via D) & 4 (via A) & 5 (via C)\\
 \hline
\end{tabular}\\

Otrzymaliśmy cykl (A,B,D,C).


\section{Zadanie 9}

\section{Zadanie 10}
Powiedzmy, że mamy pewną sieć do której należą połączone routery A i B z drogą od A do B. Zauważmy, że jeśli w wyniku wysłania komunikatu przez pewien początkowy router, do routera A dotrze $2^{\Omega (n)}$ komunikatów, to przesłanie ich wszystkich do B zajmie nam $2^{\Omega (n)}$ kroków (bo możemy wysłać tylko jeden komunikat naraz). Czyli rozesłanie tej informacji po całej sieci zajmie przynajmniej $2^{\Omega (n)}$ kroków. Skonstruujmy taką sieć. Lista kroków konstrukcji:\\\\
S $\leftarrow$ zbiór routerów należących do sieci \\
k $\leftarrow$ 0 \\
Dopóki $i \leq n$:
\begin{itemize}
\item k $\leftarrow$ k+1
\item Utwórz router o numerze k
\item Dla każdego j $\in$ S dodaj połączenie z j do k
\item Dodaj k do S
\end{itemize}
Skonstruowana sieć ma tą własność, że dla j-tego routera wszystkie połączenia wychodzące z j do k spełniają własność $k \in [j+1, n]$. Z kolei wszystkie połączenia wchodzące z k do j spełaniają własność $k \in [1, j-1]$. Pierwsze komunikaty zostają wysłane przez router 1.
Oznaczmy przez $S_k$ wynik algorytmu po k-tej iteracji. Zauważmy, że liczba komunikatów $f_1 $dla $S_1$, która dojdzie do routera 1 wynosi 0. Liczba komunikatów $f_2$ dla $S_2$ oraz 2 wynosi 1. Dla $S_3$ liczba komuniaktów które dostanie 3 jest równa liczbie komunikatów, które dochodzą do 1 oraz 2 (zostaną przekazane dalej do 3) plus komunikat od 1, czyli $f_3 = 1 + f_1 + f_2$. Dalej dostaniemy $f_4 = 1 + f_1 + f_2 + f_3$, ... \\
Możemy stąd wyznaczyć wzór $f_n = 1 + \sum_{i=0}^{n-1} f_i$. Przekształćmy wzór:\\
\( f_n = 1 + \sum_{i=0}^{n-1} f_i = 1 + \sum_{i=0}^{n-2} f_i + f_{n-1} = (1 + \sum_{i=0}^{n-2} f_i) + f_{n-1} = f_{n-1} + f_{n-1} = 2\cdot f_{n-1} \).\\
Zauważając, że $f_2 = 1$ dostajemy $f_{n} = 2^{n-2}$. Czyli do n-tego routera trafia w naszej sieci $2^{n-2}$ komunikatów.
Potrzebujemy jednego oddzielnego routera, do którego nasz n-ty router będzie przesyłał te komunikaty, więc zabierając zmniejszająć liczbę routerów w S o 1 dostajemy $2^{n-3}$ komunikatów dochodzących do n-tego routera. Przesłanie tych komunikatów do tego oddzielnego wyróżnionego zajmie $2^{n-3}$. Wysyłanie tych komunikatów zacznie się, gdy dostaniemy pierwszy komunikat od routera 1 i będzie trwało to aż do wysłania ostatniego z kolejki. Oczywiście ta operacja będzie wykonywała się najdłużej ze wszystkich routerów, stąd przesłanie informacji zakończy się po czasie $2^{\Omega (n)}$.


\end{document}